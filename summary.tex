\section{Summary}
Elastic full waveform inversion (EFWI) provides high-resolution parameter estimation of the
subsurface but requires good initial guess of the true model.
%to cope with the serious
%nonlinearity and parameter trade-offs.
%Elastic reflections carry background information of the P and S wave velocities can be in favor of EFWI with good initial P and S wave
%velocity models, 
The traveltime inversion only minimizes traveltime misfits which are more sensitive and
linearly related to the low-wavenumber model perturbation.
Therefore, building initial P and S wave velocity models for EFWI by using elastic wave-equation
reflections traveltime inversion (WERTI) would be effective and robust, especially for the deeper part. 
%However, in elastic case, it is
%difficult to extract the traveltime of a particular wave modes due to complicated
%mode-conversions.
In order to distinguish the reflection travletimes of P or S-waves in elastic media,
we decompose the surface multicomponent data into vector P- and S-wave seismogram.
We utilize the dynamic image warping to extract the reflected P- or S-wave
traveltimes.
The P-wave velocity are first inverted using P-wave traveltime followed by the S-wave velocity
inversion with S-wave traveltime,
during which the wave mode decomposition is applied to the gradients calculation.
Synthetic example on the Sigbee2A model proves the validity
of our method for recovering the long wavelength components of the model.
\onecolumn
