\section{Introduction}
Elastic full
waveform inversion provides high-resolution 
model estimation but notoriously surfers from the nonlinearities of
multi-parameter inversion and also the same cycle-skipping problem as in acoustic
case \cite[]{sears:2008,brossier2009}.
\cite{xu:2012} suggested using a reflection waveform inversion (RWI) method, 
which aim to
invert the long-wavelength components of the model by using the reflections 
predicted by migration/demigration process.
Through minimizing misfit function of waveform, the RWI method are developed
by several work \cite[]{Wu2015b,Zhou2015}, and recently extended to elastic case by
\cite{Guo2016}.

%\cite{Zhou2015} proposed a joint FWI method to combine the diving and reflected waves to
%utilize both the conventional FWI and RWI.
%In addition, \cite{Wang:2015} found that wave mode decomposition may be beneficial to deal with the
%elastic parameter trade-offs.
However, traveltime information are more sensitive and linearly related to
low-wavenumber model perturbation. Using traveltime inversion will be more robust and helpful to
build appropriate initial models containing long-wavelength components for
conventional FWI \cite[]{Ma2013, Chi2015, Luo2016}.
Unfortunately, in elastic case, traveltimes of a particular wave modes are difficult
to extract due to the complicated mode-conversions.
%Elastic reflections carry background information of the P and S wave velocities can
%be in favor of EFWI with good initial P and S wave
%velocity models, 
In this abstract,   
we aim to tackle the traveltime misfits of P-P and P-S reflections with the help
of wave mode decomposition and dynamic image warping (DIW) \cite[]{Hale2013}.
Then, we use the traveltime of P-P and P-S reflections to implement the 
WERTI method \cite[]{Ma2013} with a two-stage workflow.
Finally, the numerical example of Sigsee2A model proves the robustness and validity of our
elastic WERTI method.
